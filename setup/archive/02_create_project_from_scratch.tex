% Options for packages loaded elsewhere
% Options for packages loaded elsewhere
\PassOptionsToPackage{unicode}{hyperref}
\PassOptionsToPackage{hyphens}{url}
\PassOptionsToPackage{dvipsnames,svgnames,x11names}{xcolor}
%
\documentclass[
  letterpaper,
  DIV=11,
  numbers=noendperiod]{scrartcl}
\usepackage{xcolor}
\usepackage{amsmath,amssymb}
\setcounter{secnumdepth}{-\maxdimen} % remove section numbering
\usepackage{iftex}
\ifPDFTeX
  \usepackage[T1]{fontenc}
  \usepackage[utf8]{inputenc}
  \usepackage{textcomp} % provide euro and other symbols
\else % if luatex or xetex
  \usepackage{unicode-math} % this also loads fontspec
  \defaultfontfeatures{Scale=MatchLowercase}
  \defaultfontfeatures[\rmfamily]{Ligatures=TeX,Scale=1}
\fi
\usepackage{lmodern}
\ifPDFTeX\else
  % xetex/luatex font selection
\fi
% Use upquote if available, for straight quotes in verbatim environments
\IfFileExists{upquote.sty}{\usepackage{upquote}}{}
\IfFileExists{microtype.sty}{% use microtype if available
  \usepackage[]{microtype}
  \UseMicrotypeSet[protrusion]{basicmath} % disable protrusion for tt fonts
}{}
\makeatletter
\@ifundefined{KOMAClassName}{% if non-KOMA class
  \IfFileExists{parskip.sty}{%
    \usepackage{parskip}
  }{% else
    \setlength{\parindent}{0pt}
    \setlength{\parskip}{6pt plus 2pt minus 1pt}}
}{% if KOMA class
  \KOMAoptions{parskip=half}}
\makeatother
% Make \paragraph and \subparagraph free-standing
\makeatletter
\ifx\paragraph\undefined\else
  \let\oldparagraph\paragraph
  \renewcommand{\paragraph}{
    \@ifstar
      \xxxParagraphStar
      \xxxParagraphNoStar
  }
  \newcommand{\xxxParagraphStar}[1]{\oldparagraph*{#1}\mbox{}}
  \newcommand{\xxxParagraphNoStar}[1]{\oldparagraph{#1}\mbox{}}
\fi
\ifx\subparagraph\undefined\else
  \let\oldsubparagraph\subparagraph
  \renewcommand{\subparagraph}{
    \@ifstar
      \xxxSubParagraphStar
      \xxxSubParagraphNoStar
  }
  \newcommand{\xxxSubParagraphStar}[1]{\oldsubparagraph*{#1}\mbox{}}
  \newcommand{\xxxSubParagraphNoStar}[1]{\oldsubparagraph{#1}\mbox{}}
\fi
\makeatother

\usepackage{color}
\usepackage{fancyvrb}
\newcommand{\VerbBar}{|}
\newcommand{\VERB}{\Verb[commandchars=\\\{\}]}
\DefineVerbatimEnvironment{Highlighting}{Verbatim}{commandchars=\\\{\}}
% Add ',fontsize=\small' for more characters per line
\usepackage{framed}
\definecolor{shadecolor}{RGB}{241,243,245}
\newenvironment{Shaded}{\begin{snugshade}}{\end{snugshade}}
\newcommand{\AlertTok}[1]{\textcolor[rgb]{0.68,0.00,0.00}{#1}}
\newcommand{\AnnotationTok}[1]{\textcolor[rgb]{0.37,0.37,0.37}{#1}}
\newcommand{\AttributeTok}[1]{\textcolor[rgb]{0.40,0.45,0.13}{#1}}
\newcommand{\BaseNTok}[1]{\textcolor[rgb]{0.68,0.00,0.00}{#1}}
\newcommand{\BuiltInTok}[1]{\textcolor[rgb]{0.00,0.23,0.31}{#1}}
\newcommand{\CharTok}[1]{\textcolor[rgb]{0.13,0.47,0.30}{#1}}
\newcommand{\CommentTok}[1]{\textcolor[rgb]{0.37,0.37,0.37}{#1}}
\newcommand{\CommentVarTok}[1]{\textcolor[rgb]{0.37,0.37,0.37}{\textit{#1}}}
\newcommand{\ConstantTok}[1]{\textcolor[rgb]{0.56,0.35,0.01}{#1}}
\newcommand{\ControlFlowTok}[1]{\textcolor[rgb]{0.00,0.23,0.31}{\textbf{#1}}}
\newcommand{\DataTypeTok}[1]{\textcolor[rgb]{0.68,0.00,0.00}{#1}}
\newcommand{\DecValTok}[1]{\textcolor[rgb]{0.68,0.00,0.00}{#1}}
\newcommand{\DocumentationTok}[1]{\textcolor[rgb]{0.37,0.37,0.37}{\textit{#1}}}
\newcommand{\ErrorTok}[1]{\textcolor[rgb]{0.68,0.00,0.00}{#1}}
\newcommand{\ExtensionTok}[1]{\textcolor[rgb]{0.00,0.23,0.31}{#1}}
\newcommand{\FloatTok}[1]{\textcolor[rgb]{0.68,0.00,0.00}{#1}}
\newcommand{\FunctionTok}[1]{\textcolor[rgb]{0.28,0.35,0.67}{#1}}
\newcommand{\ImportTok}[1]{\textcolor[rgb]{0.00,0.46,0.62}{#1}}
\newcommand{\InformationTok}[1]{\textcolor[rgb]{0.37,0.37,0.37}{#1}}
\newcommand{\KeywordTok}[1]{\textcolor[rgb]{0.00,0.23,0.31}{\textbf{#1}}}
\newcommand{\NormalTok}[1]{\textcolor[rgb]{0.00,0.23,0.31}{#1}}
\newcommand{\OperatorTok}[1]{\textcolor[rgb]{0.37,0.37,0.37}{#1}}
\newcommand{\OtherTok}[1]{\textcolor[rgb]{0.00,0.23,0.31}{#1}}
\newcommand{\PreprocessorTok}[1]{\textcolor[rgb]{0.68,0.00,0.00}{#1}}
\newcommand{\RegionMarkerTok}[1]{\textcolor[rgb]{0.00,0.23,0.31}{#1}}
\newcommand{\SpecialCharTok}[1]{\textcolor[rgb]{0.37,0.37,0.37}{#1}}
\newcommand{\SpecialStringTok}[1]{\textcolor[rgb]{0.13,0.47,0.30}{#1}}
\newcommand{\StringTok}[1]{\textcolor[rgb]{0.13,0.47,0.30}{#1}}
\newcommand{\VariableTok}[1]{\textcolor[rgb]{0.07,0.07,0.07}{#1}}
\newcommand{\VerbatimStringTok}[1]{\textcolor[rgb]{0.13,0.47,0.30}{#1}}
\newcommand{\WarningTok}[1]{\textcolor[rgb]{0.37,0.37,0.37}{\textit{#1}}}

\usepackage{longtable,booktabs,array}
\usepackage{calc} % for calculating minipage widths
% Correct order of tables after \paragraph or \subparagraph
\usepackage{etoolbox}
\makeatletter
\patchcmd\longtable{\par}{\if@noskipsec\mbox{}\fi\par}{}{}
\makeatother
% Allow footnotes in longtable head/foot
\IfFileExists{footnotehyper.sty}{\usepackage{footnotehyper}}{\usepackage{footnote}}
\makesavenoteenv{longtable}
\usepackage{graphicx}
\makeatletter
\newsavebox\pandoc@box
\newcommand*\pandocbounded[1]{% scales image to fit in text height/width
  \sbox\pandoc@box{#1}%
  \Gscale@div\@tempa{\textheight}{\dimexpr\ht\pandoc@box+\dp\pandoc@box\relax}%
  \Gscale@div\@tempb{\linewidth}{\wd\pandoc@box}%
  \ifdim\@tempb\p@<\@tempa\p@\let\@tempa\@tempb\fi% select the smaller of both
  \ifdim\@tempa\p@<\p@\scalebox{\@tempa}{\usebox\pandoc@box}%
  \else\usebox{\pandoc@box}%
  \fi%
}
% Set default figure placement to htbp
\def\fps@figure{htbp}
\makeatother





\setlength{\emergencystretch}{3em} % prevent overfull lines

\providecommand{\tightlist}{%
  \setlength{\itemsep}{0pt}\setlength{\parskip}{0pt}}



 


\KOMAoption{captions}{tableheading}
\makeatletter
\@ifpackageloaded{caption}{}{\usepackage{caption}}
\AtBeginDocument{%
\ifdefined\contentsname
  \renewcommand*\contentsname{Table of contents}
\else
  \newcommand\contentsname{Table of contents}
\fi
\ifdefined\listfigurename
  \renewcommand*\listfigurename{List of Figures}
\else
  \newcommand\listfigurename{List of Figures}
\fi
\ifdefined\listtablename
  \renewcommand*\listtablename{List of Tables}
\else
  \newcommand\listtablename{List of Tables}
\fi
\ifdefined\figurename
  \renewcommand*\figurename{Figure}
\else
  \newcommand\figurename{Figure}
\fi
\ifdefined\tablename
  \renewcommand*\tablename{Table}
\else
  \newcommand\tablename{Table}
\fi
}
\@ifpackageloaded{float}{}{\usepackage{float}}
\floatstyle{ruled}
\@ifundefined{c@chapter}{\newfloat{codelisting}{h}{lop}}{\newfloat{codelisting}{h}{lop}[chapter]}
\floatname{codelisting}{Listing}
\newcommand*\listoflistings{\listof{codelisting}{List of Listings}}
\makeatother
\makeatletter
\makeatother
\makeatletter
\@ifpackageloaded{caption}{}{\usepackage{caption}}
\@ifpackageloaded{subcaption}{}{\usepackage{subcaption}}
\makeatother
\usepackage{bookmark}
\IfFileExists{xurl.sty}{\usepackage{xurl}}{} % add URL line breaks if available
\urlstyle{same}
\hypersetup{
  pdftitle={Create Python Project from Scratch},
  pdfauthor={Jeremy Mikecz},
  colorlinks=true,
  linkcolor={blue},
  filecolor={Maroon},
  citecolor={Blue},
  urlcolor={Blue},
  pdfcreator={LaTeX via pandoc}}


\title{Create Python Project from Scratch}
\author{Jeremy Mikecz}
\date{2026-01-13}
\begin{document}
\maketitle


\section{Create a Python Project from Scratch with Visual Studio Code \&
UV}\label{create-a-python-project-from-scratch-with-visual-studio-code-uv}

This document provides detailed instructions for setting up your Python
environment for this course/workshop. You should have already read the
information in the \href{00_setup-slides.html}{00\_setup\_slides} and
completed the installation of Visual Studio Code and Python using uv by
following \href{01_python-setup-instructions.qmd}{Setup Notebook 01}.

\subsection{Table of Contents}\label{table-of-contents}

\begin{enumerate}
\def\labelenumi{\arabic{enumi}.}
\tightlist
\item
  setup project in Visual Studio Code (10 mins.)
\item
  write your first markdown document (10)
\item
  set up virtual environment
\item
  write your first script
\item
  write your first notebook
\item
  import external data
\end{enumerate}

\subsection{1. Setup Project in Visual Studio
Code}\label{setup-project-in-visual-studio-code}

\subsubsection{1a. Set up sample project
folder}\label{a.-set-up-sample-project-folder}

\begin{enumerate}
\def\labelenumi{\arabic{enumi}.}
\item
  Create a new folder for your sample project in a place which you can
  easily find again. Follow recommended file naming projects (outlined
  in \href{00_setup_slides.html}{00\_setup\_slides}) such as using
  underscores \texttt{\_} in place of whitespace, avoiding special
  characters, etc. i.e.:

\begin{verbatim}
~/Users/your-user-name/Documents/python_code/sample_project
\end{verbatim}
\end{enumerate}

\subsubsection{1b. Open Project Folder in Visual Studio
Code}\label{b.-open-project-folder-in-visual-studio-code}

Visual Studio Code and other IDEs allow you to load an entire project at
once. Thus, you can easily switch between projects and when you return
to a given project all the same files that were open before are opened
once again. Below, we will open our sample project folder and modify it.

\begin{enumerate}
\def\labelenumi{\arabic{enumi}.}
\item
  Open Visual Studio Code, which you installed with
  \href{01_python_setup_instructions_uv.qmd}{notebook 01}.
\item
  Click on \textbf{File} --\textgreater{} \textbf{Open Folder}. Navigate
  to your new project folder and open it.
\item
  In the left pane of VSC, click on the \textbf{Explorer} icon (looks
  like pages or files) and you will see your empty folder there.

  \pandocbounded{\includegraphics[keepaspectratio]{images/screenshot_vsc-explorer.png}}
\item
  Add the following folders using the **+ Folder icon **:
\end{enumerate}

\begin{verbatim}
|--code
|--data
|--docs
|--results
\end{verbatim}

\pandocbounded{\includegraphics[keepaspectratio]{images/screenshot_vsc-folder-setup.png}}

\subsection{2. Create a Virtual Environment for your project in
VSC}\label{create-a-virtual-environment-for-your-project-in-vsc}

\begin{enumerate}
\def\labelenumi{\arabic{enumi}.}
\item ~
  \subsection{2. Write your first markdown
  document}\label{write-your-first-markdown-document}
\end{enumerate}

A \textbf{markdown} file is a text file that allows you to use simple
symbols to specify its formatting. This notebook and the
\textbf{00\_setup\_slides} presentation were both created using
markdown.

Some commonly used symbols:

\begin{verbatim}
# Header level 1 (note '#' should be the first symbol on the line and followed by a single whitespace)
## Header level 2
### Header level 3
**bold**
*italics*
[filename](link/to/file)   # to create hyperlinks
```
place code blocks between sets of 3 back ticks
```
\end{verbatim}

In the Explorer pane on the left, create a new file using the \textbf{+
File icon} and name it \texttt{README.md} (\textbf{.md} is the extension
for a markdown file).

Create a skeletal outline of a README file using headers of different
levels, bold, italics, links, and code blocks (see the symbol directory
above or the Markdown Guide's
\href{https://www.markdownguide.org/cheat-sheet/}{Cheat Sheet} for more
examples of Markdown syntax). Include the following information:

\begin{enumerate}
\def\labelenumi{\arabic{enumi}.}
\tightlist
\item
  tentative name of your project
\item
  name(s) of author(s) (you!)
\item
  Overview section
\item
  Data section
\item
  Methods section
\end{enumerate}

Save the file.

\subsection{3. Setting Up a Python Virtual Environment for your
project}\label{setting-up-a-python-virtual-environment-for-your-project}

As noted in the opening slides presentation
(\href{00_setup_slides.html}{html} or \href{00_setup_slides.pdf}{pdf})
``A virtual environment keeps each project's packages separate and
organized, preventing conflicts and making your code easy to share.''

We will now use \texttt{uv} to create a virtual environment for our
sample project.

\subsubsection{3a. Open the command line in
VSC}\label{a.-open-the-command-line-in-vsc}

\begin{enumerate}
\def\labelenumi{\arabic{enumi}.}
\tightlist
\item
  Select the \textbf{Terminal} tab at the top of VSC --\textgreater{}
  \textbf{New Terminal}
\item
  A new terminal should appear at the bottom.
\item
  If you are using a PC, it is recommended you use a \textbf{Command
  Prompt} terminal instead of \textbf{Powershell}. To open Command,
  select the \texttt{+} symbol at the far right side of the terminal
  pane, and select \textbf{Command Prompt}.
\end{enumerate}

\subsubsection{3b. Confirm VSC can find your Python and uv
installations}\label{b.-confirm-vsc-can-find-your-python-and-uv-installations}

\begin{enumerate}
\def\labelenumi{\arabic{enumi}.}
\tightlist
\item
  In the \textbf{terminal}, you can test to see if VSC can find your
  Python and uv installations (just as we did in notebook 01) by running
  the following code:
\end{enumerate}

\begin{verbatim}
uv --version
\end{verbatim}

\begin{verbatim}
uv python list
\end{verbatim}

\subsubsection{3c. Create Virtual
Environment}\label{c.-create-virtual-environment}

Make sure you are in the project root folder, i.e.~your terminal should
look like something like this (with a path to your project directory
shown):

\pandocbounded{\includegraphics[keepaspectratio]{images/screenshot_bash_rootdir.png}}

If you are not in the project root directory you can navigate to it by
using the \texttt{cd} (change directory) command. Some tips:

\begin{enumerate}
\def\labelenumi{\arabic{enumi}.}
\tightlist
\item
  to navigate up one directory type \texttt{cd\ ..}
\item
  to navigate up 3 directories type \texttt{cd\ ../../..}
\item
  to navigate from the current working directory into your code folder
  type \texttt{cd\ code}.
\item
  to navigate down two directories, type something like this:
  \texttt{cd\ code/scripts} (assuming the code folder is in your current
  working directory and a scripts folder is inside of that.
\end{enumerate}

Now, run the following in your project's root directory from the
\textbf{terminal}:

\begin{verbatim}
# This initiaties a new uv environment for your project
uv init
\end{verbatim}

You will see in the \textbf{Explorer pane} on the left side of VSC
several new files and folders were created in your project's working
directory. The above command:

\begin{itemize}
\item
  Installed Python if the correct version isn't available
\item
  Created a \texttt{pyproject.toml} file, where uv will store
  information about the packages / dependencies used in this project.
\item
  Created a \texttt{.python-version} file to store the version of Python
  used for this project.
\end{itemize}

Initiating \textbf{uv} \emph{may also create} a \texttt{uv.lock} file
and a \texttt{\textquotesingle{}.venv} folder if you have already
installed some packages. Let's test this out by installing a package to
our local environment at which point this file and folder will be
created:

Run in the terminal:

\begin{verbatim}
uv add requests
\end{verbatim}

Certainly, now you should see the following in your project's root
directory:

\begin{verbatim}
-   a `uv.lock` file which may be used by others (or by yourself in a different environment) to reproduce the exact same computing environment for your project.

-   a `.venv` folder where the code for all projects added to the local environment are stored
\end{verbatim}

\textbf{One final step:} every time we start a new project in VSC we
need to configure VSC to use the correct version of Python. We can do
this by selecting our Python interpreter:

\begin{enumerate}
\def\labelenumi{\arabic{enumi}.}
\tightlist
\item
  Open the ``Show and Run Command Bars'' (either select from central
  dropdown project menu at top of screen or by clicking
  \texttt{CTRL\ +\ SHIFT\ +\ P} on Windows or
  \texttt{CMD\ +\ CTRL\ +\ SPACE} on a Mac)
\item
  Search for and select \textbf{Python: Select Interpreter}
\item
  Choose the version of Python installed to your local environment. It
  should look something like this: `Python 3.14.2 (name-of-project)
  .venv/Scripts/python.exe Recommended''
\end{enumerate}

Let's view which version of Python is being used by this project. Run
the following in the \textbf{terminal}:

\begin{verbatim}
uv run python --version
\end{verbatim}

\subsection{VS Code Setup Checklist}\label{vs-code-setup-checklist}

\begin{verbatim}
 ✅ Install VS Code
 ✅ Install Python extension (ms-python.python)
 ✅ Install Jupyter extension
 ✅ Open project folder
 ✅ Select interpreter: Ctrl+Shift+P → "Python: Select Interpreter" → .venv
 ✅ Configure terminal to use uv
\end{verbatim}

\subsection{4. Write your first Python
script}\label{write-your-first-python-script}

\begin{enumerate}
\def\labelenumi{\arabic{enumi}.}
\tightlist
\item
  Create a Python script in your \textbf{code/} folder and name it
  something like this: \texttt{greeting.py}.

  \begin{enumerate}
  \def\labelenumii{\arabic{enumii}.}
  \item
    Create a new script either by navigating to the code/ folder,
    right-clicking it, and selecting \texttt{New\ File}. Be sure to give
    the file the \texttt{.py} extension.
  \item
    OR \textbf{File} --\textgreater{} \textbf{New File} and then give
    your file a name and place it in the \texttt{code} folder.

    *Note: using the correct extension is essential so the computer
    knows how to read it. A \texttt{.py} file is a plain text file
    containing Python code that can be run with software like VSC or
    from the command line.*
  \end{enumerate}
\item
  Type / paste in the following code (modifying it to make it your own)
  in the \textbf{Python script}:
\end{enumerate}

\begin{verbatim}
# Get the user's name
name = input("What's your name? ")

# print(name, "is a beautiful name.")
\end{verbatim}

\begin{enumerate}
\def\labelenumi{\arabic{enumi}.}
\setcounter{enumi}{2}
\tightlist
\item
  Save the script (important: when you run a script it will only run the
  most recently saved version).
\item
  Run your script in the \textbf{terminal} using uv:
\end{enumerate}

\begin{verbatim}
uv run code/greeting.py
\end{verbatim}

\begin{enumerate}
\def\labelenumi{\arabic{enumi}.}
\setcounter{enumi}{4}
\item
  Other options to run this file:

  \begin{enumerate}
  \def\labelenumii{\alph{enumii}.}
  \item
    \textbf{Right-click} anywhere within \textbf{greeting.py} (in VSC)
    and select \textbf{Run Python} --\textgreater{} \textbf{Run Python
    File in Terminal}
  \item
    You can also run the script by selecting the \textbf{Play icon
    button} at the top of the editor window where you have
    \texttt{greeting.py}. Note: with this method, you are NOT using uv
    to run the script. Therefore, VSC may not know which Python version
    on your computer to run or it may choose a version that doesn't
    work. If you get an error: choose the version that looks something
    like this:
    \texttt{Python\ 3.XX.X\ (sample-project)\ .\textbackslash{}.venv\textbackslash{}Scripts\textbackslash{}python.exe\ \ \ \ \ \ \ \ \ \ \ \ \ \ Workspace}.
  \end{enumerate}
\end{enumerate}

\subsubsection{Optional additions to
script:}\label{optional-additions-to-script}

add to your \textbf{Python script}:

\begin{verbatim}
# Get their favorite color
color = input("What's your favorite color? ")

# Respond based on their color choice
if color.lower() == "blue":
    print(f"Hello, {name}! Blue is a calm and peaceful color. Great choice!")
elif color.lower() == "red":
    print(f"Hello, {name}! Red is bold and energetic. I like it!")
elif color.lower() == "green":
    print(f"Hello, {name}! Green is the color of nature. Very refreshing!")
else:
    print(f"Hello, {name}! {color.title()} is a wonderful color!")
\end{verbatim}

\subsubsection{4b. Create a Script Requiring External Code from a Python
Package}\label{b.-create-a-script-requiring-external-code-from-a-python-package}

You can only do a few simple things with core Python code. Anything more
complex - from data analysis to statistics, working with images and
text, etc. - you will need to:

\begin{enumerate}
\def\labelenumi{\arabic{enumi}.}
\tightlist
\item
  install the package to your local environment using uv (do this once
  per project)
\item
  import the package into any scripts or notebooks that use it (must
  include in each code script or notebook)
\end{enumerate}

We will create a \textbf{random compliments generator} in a Python
script, which requires the use of two packages:

\begin{itemize}
\item
  \textbf{random} (which is part of the core Python installation and
  thus DOES NOT need to be installed, but DOES need to be imported) and
\item
  \textbf{rich} (which DOES NEED to be installed AND imported).
\end{itemize}

To install \textbf{rich} run the following in the \textbf{terminal}:

\begin{verbatim}
uv add rich 
\end{verbatim}

You can confirm the installation of this package to your local
environment in multiple ways.

\begin{enumerate}
\def\labelenumi{\arabic{enumi}.}
\tightlist
\item
  Open the \texttt{pyproject.toml} file and you should see it listed
  among the dependencies.
\item
  Open \texttt{uv.lock} and you can see information about the installed
  \textbf{rich} package.
\item
  Review code files for the rich package stored in
  \texttt{.venv/Lib/site-packages}.
\item
  Type \texttt{uv\ pip\ list} or \texttt{uv\ pip\ freeze} in the command
  line / terminal.
\end{enumerate}

\emph{NOTE: Never manually modify any of the above unless you know what
you are doing. These files and folders are manually updated by
\textbf{uv} when you run uv commands.}

Then paste / type the following into a \textbf{new Python script called
\texttt{compliment\_generator.py}}.

\begin{verbatim}
import random
from rich import print

# List of compliments
compliments = [
    "You're doing an amazing job learning Python! 🎉",  # \U0001F389
    "Your curiosity is inspiring! 🌟",                  # \U00002B50
    "You have great potential as a programmer! 💻",
    "Keep up the excellent work! 🚀",                   # \U0001F680"
    "You're braver than you think for trying something new! 💪"
    "✨"
]

# Get user's name
name = input("What's your name? ")

# Pick a random compliment
chosen_compliment = random.choice(compliments)

# Print with color using rich
print(f"\n[bold cyan]Hey {name}![/bold cyan]")
print(f"[green]{chosen_compliment}[/green]\n")
\end{verbatim}

Note: to acquire these and other emojis:

\begin{enumerate}
\def\labelenumi{\arabic{enumi}.}
\tightlist
\item
  on a \textbf{PC} press \texttt{Windows\ +\ .} or
  \texttt{Windows\ +\ ;}
\item
  On a Mac type \texttt{Command\ +\ Control\ +\ Space}.
\item
  On a Linux type \texttt{Ctrl\ +\ .} or \texttt{Ctrl\ +\ ;}.
\end{enumerate}

Try running this script from the command line using the same procedure
introduced in the previous step.

\subsection{5. Write your first Jupyter
Notebook}\label{write-your-first-jupyter-notebook}

\textbf{Coding notebooks} allow a user to create a more interactive,
modular code script. These notebooks allow users to mix markdown texts
(with human-readable instructions and explanations) with code cells,
which can be run one at a time or all at once. They also allow users to
view various intermediary results (i.e.~data tables or visualizations)
right within their notebook.

\textbf{Jupyter} is the most commonly used type of coding notebook for
Python.

\subsubsection{5a. Install Jupyter}\label{a.-install-jupyter}

\begin{enumerate}
\def\labelenumi{\arabic{enumi}.}
\tightlist
\item
  To use Jupyter first we need to install it using \textbf{uv}. Run the
  following in the \textbf{terminal}:
\end{enumerate}

\begin{verbatim}
uv add jupyter
\end{verbatim}

\subsubsection{5b. Create a New
Notebook}\label{b.-create-a-new-notebook}

\begin{enumerate}
\def\labelenumi{\arabic{enumi}.}
\item
  \textbf{Right-click} in the Explorer sidebar (where your files are)
\item
  Click \textbf{``New File''}
\item
  Name it: \texttt{my\_first\_notebook.ipynb}
\end{enumerate}

\begin{verbatim}
-   The `.ipynb` extension is important!
\end{verbatim}

\begin{enumerate}
\def\labelenumi{\arabic{enumi}.}
\setcounter{enumi}{3}
\tightlist
\item
  Press Enter
\end{enumerate}

\subsubsection{5c. Select Your Python
Environment}\label{c.-select-your-python-environment}

\begin{enumerate}
\def\labelenumi{\arabic{enumi}.}
\item
  When the notebook opens, look at the \textbf{top-right corner}
\item
  Click \textbf{``Select Kernel''}
\item
  Choose \textbf{``Python Environments\ldots{}''}
\item
  Select the one that says \textbf{\texttt{.venv}} or
  \textbf{\texttt{uv}}
\end{enumerate}

\subsubsection{5d. Start Coding}\label{d.-start-coding}

\begin{enumerate}
\def\labelenumi{\arabic{enumi}.}
\item
  Click in the first cell
\item
  Type some Python code
\item
  Press \textbf{Shift + Enter} to run the cell
\item
  That's it! 🎉
\end{enumerate}

\subsubsection{5e. Fun Practice Notebook
Contents}\label{e.-fun-practice-notebook-contents}

A short cheatsheet for working with Jupyter notebooks (or click
\href{https://media.datacamp.com/legacy/image/upload/v1676302533/Marketing/Blog/Jupyterlab_Cheat_Sheet.pdf}{here}
for a more extensive cheatsheet:

\begin{verbatim}
╔══════════════════════════════════════════════╗
║      JUPYTER NOTEBOOK SHORTCUTS              ║
╠══════════════════════════════════════════════╣
║                                              ║
║  RUN CELL:                                   ║
║  • Shift + Enter  (run and move to next)     ║
║  • Ctrl + Enter   (run and stay)             ║
║                                              ║
║  ADD CELLS:                                  ║
║  • Click "+ Code" or "+ Markdown" buttons    ║
║  • Or use the + icon at top                  ║
║                                              ║
║  CHANGE CELL TYPE:                           ║
║  • Use dropdown: "Code" or "Markdown"        ║
║                                              ║
║  DELETE CELL:                                ║
║  • Click trash icon on left of cell          ║
║                                              ║
║  SAVE NOTEBOOK:                              ║
║  • Ctrl + S (or Cmd + S on Mac)              ║
║                                              ║
║  RESTART KERNEL:                             ║
║  • Click "Restart" icon at top               ║
║  • Use when things get stuck                 ║
║                                              ║
╚══════════════════════════════════════════════╝
\end{verbatim}

Copy and paste these cells into your notebook to practice:

\paragraph{📝 Cell 1: Markdown
(Introduction)}\label{cell-1-markdown-introduction}

\textbf{Click the cell, then change it to ``Markdown'' using the
dropdown at the top}

\begin{Shaded}
\begin{Highlighting}[]
\NormalTok{\# 🎉 My First Jupyter Notebook}

\NormalTok{Welcome to my coding adventure! Today I\textquotesingle{}m learning about Jupyter notebooks.}

\NormalTok{\#\# What I\textquotesingle{}ll practice:}
\NormalTok{{-} Running Python code in cells}
\NormalTok{{-} Using **markdown** for notes}
\NormalTok{{-} Making calculations}
\NormalTok{{-} Creating simple visualizations}
\NormalTok{{-}{-}{-}}

\NormalTok{**Date:** January 2025  }
\NormalTok{**Mood:** Excited! 🚀}
\end{Highlighting}
\end{Shaded}

\paragraph{Cell 2: Code (Simple Math)}\label{cell-2-code-simple-math}

\begin{Shaded}
\begin{Highlighting}[]
\CommentTok{\# Let\textquotesingle{}s do some calculations!}
\BuiltInTok{print}\NormalTok{(}\StringTok{"🧮 Quick Math Practice"}\NormalTok{)}
\BuiltInTok{print}\NormalTok{(}\StringTok{"{-}"} \OperatorTok{*} \DecValTok{30}\NormalTok{)}

\NormalTok{my\_age }\OperatorTok{=} \DecValTok{25}  \CommentTok{\# Change this to your age!}
\NormalTok{python\_age }\OperatorTok{=} \DecValTok{2025} \OperatorTok{{-}} \DecValTok{1991}  \CommentTok{\# Python was created in 1991}

\BuiltInTok{print}\NormalTok{(}\SpecialStringTok{f"I am }\SpecialCharTok{\{}\NormalTok{my\_age}\SpecialCharTok{\}}\SpecialStringTok{ years old"}\NormalTok{)}
\BuiltInTok{print}\NormalTok{(}\SpecialStringTok{f"Python is }\SpecialCharTok{\{}\NormalTok{python\_age}\SpecialCharTok{\}}\SpecialStringTok{ years old"}\NormalTok{)}
\BuiltInTok{print}\NormalTok{(}\SpecialStringTok{f"Python is }\SpecialCharTok{\{}\NormalTok{python\_age }\OperatorTok{{-}}\NormalTok{ my\_age}\SpecialCharTok{\}}\SpecialStringTok{ years older than me!"}\NormalTok{)}
\end{Highlighting}
\end{Shaded}

\begin{verbatim}
🧮 Quick Math Practice
------------------------------
I am 25 years old
Python is 34 years old
Python is 9 years older than me!
\end{verbatim}

\paragraph{📝 Cell 3: Markdown
(Observations)}\label{cell-3-markdown-observations}

\begin{Shaded}
\begin{Highlighting}[]
\NormalTok{\#\# 💭 My Observations}

\NormalTok{The cool thing about notebooks is that I can:}
\NormalTok{1. Write code}
\NormalTok{2. See the results immediately}
\NormalTok{3. Add notes about what I learned}

\NormalTok{\textgreater{} **Tip:** Press \textasciigrave{}Shift + Enter\textasciigrave{} to run a cell!}
\end{Highlighting}
\end{Shaded}

\subsubsection{5f. Create a basic
visualization}\label{f.-create-a-basic-visualization}

Let's experiment by creating a data visualization using some popular
Python packages.

\begin{enumerate}
\def\labelenumi{\arabic{enumi}.}
\tightlist
\item
  Create a new notebook and name it \texttt{plotting\_penguins.ipynb} or
  something like that.
\item
  Next, we need to install the following Python libraries / packages:

  \begin{itemize}
  \item
    \textbf{pandas} - for working with data tables
  \item
    \textbf{matplotlib} - for creating data visualizations
  \item
    \textbf{seaborn} - some additional functionality and assistance
    working with matplotlib
  \end{itemize}
\end{enumerate}

To install these, we will use \textbf{uv} in the terminal:

\begin{verbatim}
uv add pandas matplotlib seaborn
\end{verbatim}

\emph{Note: we can install multiple packages at once using this syntax.}

\begin{enumerate}
\def\labelenumi{\arabic{enumi}.}
\setcounter{enumi}{2}
\tightlist
\item
  In the first code cell in the new notebook, let's install the
  necessary packages:
\end{enumerate}

\begin{Shaded}
\begin{Highlighting}[]
\ImportTok{import}\NormalTok{ seaborn }\ImportTok{as}\NormalTok{ sns}
\ImportTok{import}\NormalTok{ matplotlib.pyplot }\ImportTok{as}\NormalTok{ plt}
\end{Highlighting}
\end{Shaded}

\begin{enumerate}
\def\labelenumi{\arabic{enumi}.}
\setcounter{enumi}{3}
\tightlist
\item
  Next, we can import a preloaded dataset from \textbf{seaborn} and
  preview its data:
\end{enumerate}

\begin{Shaded}
\begin{Highlighting}[]
\CommentTok{\# Load a built{-}in dataset (no files needed!)}
\NormalTok{penguins }\OperatorTok{=}\NormalTok{ sns.load\_dataset(}\StringTok{\textquotesingle{}penguins\textquotesingle{}}\NormalTok{)}

\CommentTok{\# Take a quick look at the data}
\BuiltInTok{print}\NormalTok{(}\StringTok{"🐧 Palmer Penguins Dataset"}\NormalTok{)}
\BuiltInTok{print}\NormalTok{(}\StringTok{"="} \OperatorTok{*} \DecValTok{50}\NormalTok{)}
\BuiltInTok{print}\NormalTok{(penguins.head())}
\BuiltInTok{print}\NormalTok{(}\SpecialStringTok{f"}\CharTok{\textbackslash{}n}\SpecialStringTok{Total penguins: }\SpecialCharTok{\{}\BuiltInTok{len}\NormalTok{(penguins)}\SpecialCharTok{\}}\SpecialStringTok{"}\NormalTok{)}
\BuiltInTok{print}\NormalTok{(}\SpecialStringTok{f"Species: }\SpecialCharTok{\{}\NormalTok{penguins[}\StringTok{\textquotesingle{}species\textquotesingle{}}\NormalTok{]}\SpecialCharTok{.}\NormalTok{unique()}\SpecialCharTok{\}}\SpecialStringTok{"}\NormalTok{)}
\end{Highlighting}
\end{Shaded}

\begin{verbatim}
🐧 Palmer Penguins Dataset
==================================================
  species     island  bill_length_mm  bill_depth_mm  flipper_length_mm  \
0  Adelie  Torgersen            39.1           18.7              181.0   
1  Adelie  Torgersen            39.5           17.4              186.0   
2  Adelie  Torgersen            40.3           18.0              195.0   
3  Adelie  Torgersen             NaN            NaN                NaN   
4  Adelie  Torgersen            36.7           19.3              193.0   

   body_mass_g     sex  
0       3750.0    Male  
1       3800.0  Female  
2       3250.0  Female  
3          NaN     NaN  
4       3450.0  Female  

Total penguins: 344
Species: ['Adelie' 'Chinstrap' 'Gentoo']
\end{verbatim}

\begin{enumerate}
\def\labelenumi{\arabic{enumi}.}
\setcounter{enumi}{4}
\tightlist
\item
  Now, let's create a pair plot of this data:
\end{enumerate}

\begin{Shaded}
\begin{Highlighting}[]
\CommentTok{\# Create visualization in ONE line!}
\NormalTok{sns.pairplot(penguins, hue}\OperatorTok{=}\StringTok{\textquotesingle{}species\textquotesingle{}}\NormalTok{, palette}\OperatorTok{=}\StringTok{\textquotesingle{}husl\textquotesingle{}}\NormalTok{)}

\NormalTok{plt.suptitle(}\StringTok{\textquotesingle{}🐧 Penguin Data Overview\textquotesingle{}}\NormalTok{, y}\OperatorTok{=}\FloatTok{1.01}\NormalTok{, fontsize}\OperatorTok{=}\DecValTok{16}\NormalTok{, fontweight}\OperatorTok{=}\StringTok{\textquotesingle{}bold\textquotesingle{}}\NormalTok{)}
\NormalTok{plt.show()}
\end{Highlighting}
\end{Shaded}

\begin{verbatim}
C:\Users\F0040RP\Documents\DartLib_RDS\projects\python-setup\.venv\Lib\site-packages\IPython\core\pylabtools.py:170: UserWarning: Glyph 128039 (\N{PENGUIN}) missing from font(s) DejaVu Sans.
  fig.canvas.print_figure(bytes_io, **kw)
\end{verbatim}

\pandocbounded{\includegraphics[keepaspectratio]{02_create_project_from_scratch_files/figure-pdf/cell-5-output-2.pdf}}

Or, we can create a scatterplot with additional customization:

\begin{Shaded}
\begin{Highlighting}[]
\CommentTok{\# Create a colorful visualization}
\NormalTok{sns.set\_style(}\StringTok{"whitegrid"}\NormalTok{)}
\NormalTok{plt.figure(figsize}\OperatorTok{=}\NormalTok{(}\DecValTok{10}\NormalTok{, }\DecValTok{6}\NormalTok{))}

\CommentTok{\# Scatter plot: flipper length vs body mass, colored by species}
\NormalTok{sns.scatterplot(}
\NormalTok{    data}\OperatorTok{=}\NormalTok{penguins,}
\NormalTok{    x}\OperatorTok{=}\StringTok{\textquotesingle{}flipper\_length\_mm\textquotesingle{}}\NormalTok{,}
\NormalTok{    y}\OperatorTok{=}\StringTok{\textquotesingle{}body\_mass\_g\textquotesingle{}}\NormalTok{,}
\NormalTok{    hue}\OperatorTok{=}\StringTok{\textquotesingle{}species\textquotesingle{}}\NormalTok{,}
\NormalTok{    size}\OperatorTok{=}\StringTok{\textquotesingle{}body\_mass\_g\textquotesingle{}}\NormalTok{,}
\NormalTok{    sizes}\OperatorTok{=}\NormalTok{(}\DecValTok{50}\NormalTok{, }\DecValTok{400}\NormalTok{),}
\NormalTok{    alpha}\OperatorTok{=}\FloatTok{0.7}\NormalTok{,}
\NormalTok{    palette}\OperatorTok{=}\StringTok{\textquotesingle{}Set2\textquotesingle{}}
\NormalTok{)}
\end{Highlighting}
\end{Shaded}

\pandocbounded{\includegraphics[keepaspectratio]{02_create_project_from_scratch_files/figure-pdf/cell-6-output-1.pdf}}

\section{Brief Cheatsheet}\label{brief-cheatsheet}

\emph{the brief, no explanation need version:}

\begin{enumerate}
\def\labelenumi{\arabic{enumi}.}
\item
  Set up a sample project folder,
  i.e.~\texttt{\textasciitilde{}/Users/your-user-name/Documents/python\_code/sample\_project}
\item
  Open VSC and set up a link to this project folder, by selecting
  \textbf{File} --\textgreater{} \textbf{Open Folder} and open the
  folder you created above.
\item
  In Explorers pane, create the following subfolders in your project
  folder: \texttt{code}, \texttt{data}, \texttt{docs}, \texttt{results}.
\item
  Create a \texttt{README.md} file with important explanatory
  information about your project. Use markdown syntax to format the
  text.
\item
  Set up a virtual environment for your project

  \begin{enumerate}
  \def\labelenumii{\arabic{enumii}.}
  \item
    Terminal (Mac/Linux or Windows Command): \texttt{uv\ -\/-version}
  \item
    Create virtual environment: \texttt{uv\ init}
  \end{enumerate}
\item
  Create a Python script in your \texttt{code} folder, giving it the
  extension: \texttt{.py}.
\item
  Select the Python Interpreter you want to use for this project by
  opening the ``Show and Run Command Bars'' (Windows:
  \texttt{CTRL\ +\ SHIFT\ +\ P}; Mac: \texttt{CMD\ +\ CTRL\ +\ SPACE}),
  select \textbf{Python: Select Interpreter}, and choose the version of
  Python installed to your local environment. I.e.: `Python 3.14.2
  (name-of-project) .venv/Scripts/python.exe Recommended''
\item
  Confirm \texttt{uv} is using the correct environment and Python:

  \begin{enumerate}
  \def\labelenumii{\alph{enumii}.}
  \tightlist
  \item
    Windows: \texttt{uv\ run\ where.exe\ python} \# first response
    should be your local Python
  \item
    Mac: \texttt{uv\ run\ which\ python}\\
  \item
    \emph{Note}: compare to the results if you run these commands
    without \texttt{uv\ run}. You can also check by comparing the
    results of \texttt{uv\ run\ python\ -\/-version} and
    \texttt{python\ -\/-version}.
  \end{enumerate}
\item
  Write code (see Step 4 above for an example script) and save your
  script.
\item
  Run your script with \texttt{uv\ run\ code/script-name.py} or by
  pressing the ``Play'' button.
\item
  To run code requiring external packages:

  \begin{enumerate}
  \def\labelenumii{\arabic{enumii}.}
  \item
    install packages with \texttt{uv\ add\ package\_name}.
  \item
    when needed, import a package with the code
    \texttt{import\ package\_name} placed at the top of the document.
  \end{enumerate}
\item
  To create a Jupyter notebook:

  \begin{enumerate}
  \def\labelenumii{\arabic{enumii}.}
  \item
    install Jupyter via the Terminal with \texttt{uv\ add\ jupyter}.
  \item
    Create a new Jupyter notebook in your \texttt{code}folder (with the
    extension \texttt{.ipynb}).
  \item
    select your Python environment: top-right corner --\textgreater{}
    \textbf{Select Kernel} --\textgreater{} Choose ``Python
    Environments'' --\textgreater{} select the one that says
    \texttt{.venv} or \texttt{uv}.
  \item
    See Steps 5e and 5f above for sample code.
  \end{enumerate}
\end{enumerate}

\subsection{Troubleshooting Section}\label{troubleshooting-section}

\subsubsection{Problem 1: ``Select Kernel'' Button
Missing}\label{problem-1-select-kernel-button-missing}

\textbf{Solution:}

\begin{itemize}
\item
  Make sure you saved the file with \texttt{.ipynb} extension
\item
  Close and reopen the file
\item
  Install the Jupyter extension: Click Extensions (sidebar) → search
  ``Jupyter'' → Install
\end{itemize}

\subsubsection{Problem 2: Kernel Won't
Start}\label{problem-2-kernel-wont-start}

\textbf{Solution:}

\begin{itemize}
\item
  Open Terminal and run: \texttt{uv\ add\ ipykernel}
\item
  Reload VS Code: Press \texttt{Ctrl+Shift+P}, type ``Reload Window'',
  press Enter
\item
  Try selecting the kernel again
\end{itemize}

\subsubsection{Problem 3: Code Cell Won't
Run}\label{problem-3-code-cell-wont-run}

\textbf{Symptoms:} Nothing happens when you press Shift+Enter

\textbf{Solution:}

\begin{itemize}
\item
  Check if kernel is connected (top-right should show Python version)
\item
  Click the kernel name and select it again
\item
  Restart kernel: Click ``Restart'' icon at the top
\end{itemize}

\subsubsection{Problem 4: Matplotlib Charts Don't
Show}\label{problem-4-matplotlib-charts-dont-show}

\textbf{Solution:}

\begin{itemize}
\item
  Make sure you ran the installation cell (Cell 6)
\item
  Add this line before your plotting code:

\begin{Shaded}
\begin{Highlighting}[]
\OperatorTok{\%}\NormalTok{matplotlib inline}
\end{Highlighting}
\end{Shaded}

\begin{verbatim}
Restart the kernel and run all cells again
\end{verbatim}
\end{itemize}

Problem 5: Input() Not Working

Symptoms: Input prompt doesn't appear or freezes

Solution:

\begin{verbatim}
This is normal in Jupyter! Look for the input box at the TOP of the cell output
Make sure you're running the cell (not just clicking it)
If stuck, click "Interrupt" button at top, then try again
\end{verbatim}

Problem 6: Can't Install Packages

Error: ``uv: command not found'' or similar

Solution:

\begin{verbatim}
Use pip instead:

::: {.cell execution_count=7}
``` {.python .cell-code}
pip install matplotlib
```

::: {.cell-output .cell-output-stdout}
```
Note: you may need to restart the kernel to use updated packages.
```
:::

::: {.cell-output .cell-output-stderr}
```
C:\Users\F0040RP\Documents\DartLib_RDS\projects\python-setup\.venv\Scripts\python.exe: No module named pip
```
:::
:::

\end{verbatim}

\subsubsection{Problem 5: Input() Not
Working}\label{problem-5-input-not-working}

\textbf{Symptoms:} Input prompt doesn't appear or freezes

\textbf{Solution:}

\begin{itemize}
\item
  This is normal in Jupyter! Look for the input box at the TOP of the
  cell output
\item
  Make sure you're running the cell (not just clicking it)
\item
  If stuck, click ``Interrupt'' button at top, then try again
\end{itemize}

\subsubsection{Problem 6: Markdown Cell Shows Code Instead of Formatted
Text}\label{problem-6-markdown-cell-shows-code-instead-of-formatted-text}

\textbf{Solution:}

\begin{itemize}
\item
  Make sure the cell type is set to ``Markdown'' (check dropdown at top)
\item
  Run the cell with Shift+Enter to render it
\item
  If still showing code, change to Markdown and run again
\end{itemize}

\subsubsection{Problem 7: Notebook is Slow or
Unresponsive}\label{problem-7-notebook-is-slow-or-unresponsive}

\textbf{Solution:}

\begin{itemize}
\item
  Restart the kernel: Click ``Restart'' icon at top
\item
  Clear all outputs: Click ``\ldots{}'' menu → ``Clear All Outputs''
\item
  Close other programs to free up memory
\item
  Save and reload VS Code
\end{itemize}




\end{document}
